\chapter{How to use this template}

\begin{itemize}
    \item Please do not touch anything in the \texttt{sty} folder. If you have any problems, or you have any feature requests, please contact Gilles Callebaut via the issue page: \url{https://github.com/AMBIENT-6G/Deliverable-Template/issues}.
    \item All relevant information should be included in the \texttt{main.tex}.
    \item If you think additional packages or definitions need to be included in the general template, also use the issue page or make a pull request.

    \item See \cref{ch:deliverable-guidelines} for guidelines on making a coherent document using this template.
\end{itemize}


\chapter{Examples for a coherent document}\label{ch:deliverable-guidelines}
\contents{Revision done}{KUL}


\begin{itemize}
    \item Add yourself to the contributor list in \texttt{input/contributors.tex}
    \item Make use of the \textbf{glossaries} package to conveniently expand abbreviations in the correct manner. The abbreviations are located in \texttt{abbr.tex}.
    % \item Use \textbf{booktabs} (Gilles can do this in the first revision phase, if need be)
    \item Use distinct and descriptive labels to reference figures/tables/\ldots. 
    \begin{itemize}
        \item Use short descriptions in the captions if you are going to include a list of figures.
        
        Usage:
        \texttt{\textbackslash caption[Short version for LoF]\{Long version to appear next to the figure\}}
    \end{itemize}
%     \item make use of short and long captions:
% \verb_\caption[this is shown in TOC]{this is shown at figure/table.}_
\item \textbf{British English} is preferred for EU projects.
\item Make use of \textit{tikzplotlib} (Python),  \textit{matlab2tikz} (Matlab) or use plain old CSV to transform eps/png/pdf figures to \textbf{tikz} figures which gives us more control over the aesthetics of the figures (coherent coloring scheme, font size, font family, size,...)
\item Try to make use of pre-defined commands to \textbf{format math notations}. These are listed in the \texttt{sty/math\_notations.tex} file. Example:
\begin{verbatim}
\usepackage{xifthen}
\newcommand{\prob}[1][]{%requires xifthen package%
\ifthenelse{\isempty{#1}}%
      {\ensuremath{P}}%
    {\ensuremath{P\left\(#1\right\)}}%
}
\newcommand{\vect}[1]{\boldsymbol{\mathrm{#1}}}
\newcommand{\mat}[1]{\boldsymbol{\mathrm{#1}}}
\newcommand{\MSE}{\mathrm{MSE}}
\newcommand{\tr}{\mathrm{tr}}
\newcommand{\moddef}{\mathrm{mod}}
\newcommand{\diag}{\mathrm{diag}}
\newcommand{\vecop}{\text{vec}}
\newcommand{\CP}{L}
\newcommand{\hddots}{\hdots}
\newcommand*{\inC}[1]{\in\mathbb{C}^{#1}}
\newcommand{\norm}[1]{\left\lVert#1\right\rVert}
\newcommand{\abs}[1]{\left\lvert#1\right\rvert}
\newcommand{\expt}[1]{\mathbb{E} \left\{#1\right\}}
\newcommand{\cn}[2]{\ensuremath{\sim\mathcal{C}\mathcal{N}\left(#1,#2\right)}}
\end{verbatim}
% \item You can use \atkul to address a question to a certain partner. Please use your own comment command, so the referenced partner, knows who asked the question, made the remark, e.g., \nxp{has a question, \attug can you elaborate?} These definitions are located in the `sty/defs.tex' file.
\end{itemize}

\section{Using Colors}
\subsection{Color theme of the project}
\begin{tabular}{ll@{\hskip 4cm}l l}
      \cellcolor{pc1}\phantom{some} &  pc1&\cellcolor{pcolor}\phantom{some} &  pcolor\\
     \cellcolor{pc2}\phantom{some} &  pc2& \cellcolor{pcolordark}\phantom{some} &  pcolordark\\
     \cellcolor{pc3}\phantom{some} &  pc3 & \cellcolor{pcolorlight}\phantom{some} &  pcolorlight\\
     \cellcolor{pc4}\phantom{some} &  pc4\\ 
     \cellcolor{pc5}\phantom{some} &  pc5\\
     \cellcolor{pc6}\phantom{some} &  pc6\\     
\end{tabular}

\subsection{Color theme of the graphs}
\begin{tabular}{ll@{\hskip 4cm}l l}
      \cellcolor{color1}\phantom{some} &  color1&\cellcolor{color4}\phantom{some} &  color4\\
     \cellcolor{color2}\phantom{some} &  color2& \cellcolor{color5}\phantom{some} &  color5\\
     \cellcolor{color3}\phantom{some} & color3 & \cellcolor{color6}\phantom{some} &  color6\\
\end{tabular}


\section{Using Tikz}
Make use of tikz whenever possible. This allows us to have coherent coloring schemes and correct scaling and naming conventions. Examples:

\begin{figure}[h]
    \centering
    \setlength{\figurewidth}{0.8\linewidth}
    \setlength{\figureheight}{0.3\linewidth}
    \input{input/00-example/example-fig}
\end{figure}

\begin{figure}[h]
    \centering
    \setlength{\figurewidth}{0.6\linewidth}
    \setlength{\figureheight}{0.3\linewidth}
    \input{input/00-example/example-fig}
\end{figure}

\FloatBarrier%
\section{Using abbreviations}










